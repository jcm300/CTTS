\documentclass{llncs}
\usepackage[utf8]{inputenc}
\usepackage{fancyvrb} 
\usepackage[portuguese]{babel}
\usepackage{ragged2e}
\usepackage{listings}

\def\changemargin#1#2{\list{}{\rightmargin#2\leftmargin#1}\item[]}
\let\endchangemargin=\endlist

\begin{document} \mainmatter
\title{Cinematographic Text Treatment System}
\titlerunning{Cinematographic Text Treatment System}
\author{José Carlos Lima Martins A78821 \and
        Miguel Miranda Quaresma A77049}
\authorrunning{José Carlos Lima Martins A78821 \and
        Miguel Miranda Quaresma A77049}
\institute{                                                                
University of Minho, Department of  Informatics, Braga, Portugal\\
e-mail: \{a78821,a77049\}@alunos.uminho.pt
}

\maketitle

\justify

\begin{abstract}

\end{abstract}

\section{Introdução}
TODO: Falar da estrutura do relatorio

\section{Análise e especificação}

\subsection{Descrição informal do problema}
No âmbito da realização de filmes, pretende-se criar um Dicionário de palavras utilizadas nesta área. Associado a cada palavra pretende-se ter: o significado; a designação comum em inglês; e uma lista de sinónimos.

O sistema de tratamento de textos cinematográficos, CTTS, deve aceitar o ficheiro onde está o dicionário (usando a linguagem por nós definida) e uma lista de ficheiros onde se encontram textos a anotar.

O objectivo do CTTS é ler o dicionário e carregá-lo para um estrutura interna de tal modo que depois possa processar os ditos textos. Para isso, deve procurar em cada um as palavras que estejam presentes no dicionário assinalando-as (sublinhado) e associando-lhes uma footnote com o termo em inglês. No fim do texto será gerado um apêndice com a lista de palavras encontradas e respectivos significados.

\subsection{Especificação do Requisitos}

\section{Concepção/desenho da Resolução}

\subsection{Gramática do Dicionário}
De modo a o utilizador definir um dicionário foi criada uma linguagem que permite associar cada palavra ao seu significado, o termo em inglês e os seus sinónimos. Sendo assim o desenho da mesma é:
\begin{lstlisting}[mathescape]
Dicionario -> LinhaDic LinhasDic '.' 
LinhasDic -> $\epsilon$ 
          | ';' LinhaDic LinhasDic 
LinhaDic -> Palavra ':' Significado ':' Palavra ':' '[' Sinonimos 
Sinonimos -> ']' 
          | Palavra ListaSin ']' 
ListaSin -> $\epsilon$ 
         | ',' Palavra ListaSin 
Palavra -> pal 
Significado -> str 
\end{lstlisting}
Portanto, a LinhaDic representa cada linha do dicionario, cada palavra bem como o seu significado, o termo em inglês e os seus sinónimos. Cada LinhaDic como em C é separada por um ``;'' sendo que a ultima linha deve terminar com um ``.'' e não um ponto e vírgula. Os parâmetros de uma linha são separados por ``:'' sendo que o primeiro é a palavra do qual queremos obter informação, o Significado é como o próprio nome diz o seu significado, a seguir temos o termo em inglês e por fim temos os sinónimos. Os sinónimos estão dentro de parenteses retos sendo divididos entre si por virgulas, como uma lista.

\subsection{Estruturas}

\subsection{Algoritmos(Lógica)}

\section{Codificação e Testes}

\section{Conclusão}

\appendix
\section{Código do Flex}
\small
\begin{changemargin}{-4cm}{-4cm}
\lstinputlisting[language=C, showstringspaces=false, breaklines]{../src/ctts.l}
\end{changemargin}

\normalsize
\section{Código do Yacc}
\small
\begin{changemargin}{-4cm}{-4cm}
\lstinputlisting[language=C, showstringspaces=false, breaklines]{../src/ctts.y}
\end{changemargin}

\end{document}
