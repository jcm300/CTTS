\documentclass{llncs}
\usepackage[utf8]{inputenc}
\usepackage{fancyvrb} 
\usepackage[portuguese]{babel}
\usepackage{ragged2e}
\usepackage{listings}
\usepackage{spverbatim}

\def\changemargin#1#2{\list{}{\rightmargin#2\leftmargin#1}\item[]}
\let\endchangemargin=\endlist

\begin{document} \mainmatter
\title{Cinematographic Text Treatment System}
\titlerunning{Cinematographic Text Treatment System}
\author{José Carlos Lima Martins A78821 \and
        Miguel Miranda Quaresma A77049}
\authorrunning{José Carlos Lima Martins A78821 \and
        Miguel Miranda Quaresma A77049}
\institute{                                                                
University of Minho, Department of  Informatics, Braga, Portugal\\
e-mail: \{a78821,a77049\}@alunos.uminho.pt
}

\maketitle

\justify

\begin{abstract}
Neste documento é apresentada várias decisões e processos tomados de modo a desenvolver um parser Flex/Yacc que processe textos tendo em conta um dicionário que respeite a gramática construída. Este processamento de textos deve identificar palavras do texto que estão presentes no dicionário e sublinha-las adicionar um footnote e no fim do texto criar um apêndice com as palavras encontradas e o seu significado, tornando este processamento um texto normal(\verb|.txt| por exemplo) num ficheiro \LaTeX(\verb|.tex|).
\end{abstract}

\section{Introdução}
Neste trabalho desenvolvemos um sistema que a partir de um dicionário de palavras cinematográficas processa ficheiros de texto criando um ficheiro \LaTeX em que as palavras que estão presentes no dicionário e no texto são sublinhadas, sendo criadas footnotes bem como um apêndice com as definições das mesmas. Apesar de o trabalho ser vocacionado para um dicionário de palavras cinematográficas, o mesmo pode ser aplicado a palavras de todas as áreas estando apenas dependente do que pretende o utilizador.

Em relação à estrutura do relatório, inicialmente será exposto, de uma forma aprofundada, o problema a resolver sendo, de seguida, demonstrada uma possível solução para o mesmo. São depois ilustrados vários exemplos de teste, finalizando com uma conclusão relativamente aos resultados do trabalho.

\section{Análise e especificação}

\subsection{Descrição informal do problema}
No âmbito da realização de filmes, pretende-se criar um Dicionário de palavras utilizadas nesta área. Associado a cada palavra pretende-se ter: o significado; a designação comum em inglês; e uma lista de sinónimos.

O sistema de tratamento de textos cinematográficos, CTTS, deve aceitar o ficheiro onde está o dicionário (usando a linguagem por nós definida) e uma lista de ficheiros onde se encontram textos a anotar.

O objectivo do CTTS é ler o dicionário e carregá-lo para um estrutura interna de tal modo que depois possa processar os ditos textos. Para isso, deve procurar em cada um as palavras que estejam presentes no dicionário assinalando-as (sublinhado) e associando-lhes uma footnote com o termo em inglês. No fim do texto será gerado um apêndice com a lista de palavras encontradas e respectivos significados.

\subsection{Especificação dos Requisitos}
O sistema desenvolvido deve permitir o uso de dicionários referentes a diversas áreas do conhecimento, não se restringindo apenas à área cinematográfica, algo que deve ser garantido através da gramática desenvolvida. Por outro lado, dado a quantidade de termos possíveis o processamento do dicionário e subsequente consulta da estrutura usada para armazenar o mesmo devem ser realizados de maneira eficiente \textbf{i.e.} em tempo útil. A ferramenta deve ainda permitir o processamento de diversos ficheiros de texto ao mesmo tempo gerando um ficheiro \LaTeX para cada um.

\section{Concepção/desenho da Resolução}

\subsection{Gramática do Dicionário}
De modo a o utilizador definir um dicionário foi criada uma linguagem que permite associar cada palavra ao seu significado, o termo em inglês e os seus sinónimos. Sendo assim o desenho da mesma é:
\begin{lstlisting}[mathescape]
Dicionario -> LinhaDic LinhasDic '.' 
LinhasDic -> $\epsilon$ 
          | ';' LinhaDic LinhasDic 
LinhaDic -> Palavra ':' Significado ':' Palavra ':' '[' Sinonimos 
Sinonimos -> ']' 
          | Palavra ListaSin ']' 
ListaSin -> $\epsilon$ 
         | ',' Palavra ListaSin 
Palavra -> pal 
Significado -> str 
\end{lstlisting}
Portanto, a LinhaDic representa cada linha do dicionario, cada palavra bem como o seu significado, o termo em inglês e os seus sinónimos. Cada LinhaDic como em C é separada por um ``;'' sendo que a ultima linha deve terminar com um ``.'' e não um ponto e vírgula. Os parâmetros de uma linha são separados por ``:'' sendo que o primeiro é a palavra do qual queremos obter informação, o Significado é como o próprio nome diz o seu significado, a seguir temos o termo em inglês e por fim temos os sinónimos. Os sinónimos estão dentro de parenteses retos sendo divididos entre si por virgulas, como uma lista.

\subsection{Estruturas}
Visto existir a necessidade de processar vários textos com o dicionário, o mesmo deve ser guardado em memória. Para tal foram criadas as seguintes estruturas:
\renewcommand{\labelitemi}{$\bullet$}
\begin{itemize}
\item Lista ligada que permite representar os sinónimos
\begin{lstlisting}
typedef struct sinonym {
    char *sinonym;
    struct sinonym *next;
}*Sinonym;
\end{lstlisting}
\item Lista ligada que permite representar as linhas do dicionário
\begin{lstlisting}
typedef struct singleTerm{
    char *term;
    char *definition;
    char *designationEN;
    int refCount;
    Sinonym sinonyms;
    struct singleTerm *next;
}*SingleTerm;
\end{lstlisting}
\end{itemize}
A inserção da variável refCount na estrutura deve-se ao facto de permitir saber ao fim de processar cada texto que palavras foram encontradas sendo que, após criado o apêndice, esse valor volta a ser colocado a 0. Este valor para futuras implementações pode ser usado para saber quantos ``matches'' houve para cada palavra do dicionário.

\subsection{Parse dos ficheiros}
Já temos a gramática do dicionário e a estrutura usada para guardar os dados do mesmo contudo falta-nos uma das partes mais importantes, realizar o parse dos ficheiros. Esta fase pode ser dividida em duas:
\begin{itemize}
\item Parse do ficheiro com o dicionário no qual durante o processo devem ser identificadas as diferentes partes da gramática, retornando o conteúdo (caso necessário) e o tipo para o Yacc. No Yacc consoante a parte da gramática em que estamos no momento é realizada certas ações, no nosso caso guardar os valores na estrutura já criada.
\item Parse dos ficheiros com os textos, nos quais de acordo com se a palavra está ou não presente na estrutura, sublinhar, criar o footnote e no fim de processar o ficheiro criar um apêndice com as palavras identificadas e os seus significados.
\end{itemize}
É por causa disto que no ficheiro Flex existem duas condições de contexto, DICT e FILES. Antes de cada iniciação do Flex é chamado uma das funções parseDict ou parseFiles para que o Flex comece respetivamente dentro da condição de contexto DICT ou FILES. Sendo assim, dentro de cada condição é colocado as regras especificas para cada tipo de ficheiro(DICT-Dicionário e FILES-ficheiro a ``converter''), permitindo assim realizar o parse de vários ficheiros com estruturas diferentes como o mesmo ficheiro Flex.

\subsection{Underline e footnote}

\subsection{Funcionalidades Adicionais}
Por forma a aumentar a utilidade da ferramenta desenvolvida foram implementadas algumas funcionalidades adicionais, nomeadamente:
\begin{itemize}
    \item Possibilidade de escolher nome de ficheiro output para cada ficheiro de texto processado através da flag \textbf{-o}
    \item Procura/comparação de termos sem ter em conta a grafia(maiúscula ou minúscula) de uma letra
\end{itemize}


\subsection{Apêndice}
Como já descrito anteriormente a partir da variável refCount da estrutura SingleTerm é possível saber se a palavra foi ou não encontrada no texto. Sendo assim basta verificar esse valor e caso seja maior que 0 imprime-se um item de uma lista com o nome do termo e a definição:
\begin{verbatim}
while(aux){
    if(aux->refCount>0){
        fprintf(f,"\\item %s $\\to$ Def: %s\n",aux->term,aux->definition);
        aux->refCount=0;
    }
    aux=aux->next;
}
\end{verbatim}
Contudo é antes disto imprimido para o ficheiro a indicação de iniciação do apêndice, criada uma secção Apendice e a abertura da lista:
\begin{verbatim}
fprintf(f,"\n\\appendix\n");
fprintf(f,"\\section{Apendice}\n");
fprintf(f,"\\begin{itemize}\n");
\end{verbatim}
No fim de gerar todos os elementos da lista, fecha-se a lista bem como o documento:
\begin{verbatim}
fprintf(f,"\\end{itemize}\n");
fprintf(f,"\n\\end{document}\n");
\end{verbatim}

\section{Codificação e Testes}
Foi criado uma Makefile de modo a compilar o código sendo que a ideia base é:
\begin{itemize}
    \item Gerar \verb|.c| do flex: \verb|flex ctts.l|
    \item Gerar \verb|.c| do yacc: \verb|yacc -d -v ctts.y|
    \item Gerar o executável: \verb|gcc -o ctts y.tab.c|
\end{itemize}

O executável pode ser usado de formas diferentes dependendo de como é chamado e que flags são passadas:
\begin{itemize}
    \item \verb|./ctts <FicheiroDicionario> <FicheiroARealizarParse1> <FicheiroArealizarParse2> ...|
    \item \verb|./ctts <FicheiroDicionario> <FicheiroARealizarParse> -o <NomeDeFicheiroDeSaida>|
\end{itemize}
É importante frisar que no primeiro caso os nomes dos ficheiros de output serão \verb|NomeDoFicheiroInput.tex| enquanto que no segundo caso será o \verb|NomeInseridoPeloUser.tex|.

\subsection{Exemplo de teste}
Comando executado: \verb|./ctts dict.txt tt.txt -o file|
\subsubsection{Input: tt.txt}
\begin{spverbatim}
Tornar-se um roteirista começa com o desenvolvimento do seu talento como escritor, e o melhor jeito de desenvolver o seu talento como escritor é praticando. O seu primeiro projeto não precisa ser um roteiro de cinema ou de TV. Você apenas precisa começar em algum lugar: ensaios, contos, peças, etc. A idéia é estar habituado com o processo de escrita - uma experiência surpreendentemente lenta e frustrante - e encontrar o seu estilo. Com o passar do tempo, você descobrirá em que gênero seu talento é melhor aproveitado: comédia, drama ou outro.
\end{spverbatim}

\subsubsection{Output: file.tex}
\begin{spverbatim}
\documentclass{article}
\usepackage[bottom]{footmisc}

\begin{document}
Tornar-se um roteirista começa com o desenvolvimento do seu talento como escritor, e o melhor jeito de desenvolver o seu talento como escritor é praticando. O seu primeiro projeto não precisa ser um             \underline{roteiro}\footnote{script} de \underline{cinema}\footnote{cinema} ou de TV. Você apenas precisa começar em algum lugar: ensaios, contos, peças, etc. A idéia é estar habituado com o processo de escrita - uma experiência surpreendentemente lenta e frustrante - e encontrar o seu estilo. Com o passar do tempo, você descobrirá em que gênero seu talento é melhor aproveitado: comédia, drama ou outro.

\appendix
\section{Apendice}
\begin{itemize}
\item roteiro $\to$ Def: a historia do filme escrita em papel. Com as falas e tudo que for pertinente para a composicao do filme.
\item quadro $\to$ Def: a imagem unica estatica do filme, e a menor unidade de um filme. Varias imagens (frames) geram ilusao de movimento. Em geral, o ritmo e 24 quadros por segundo. Hoje em dia nao e necessario mais gravar em pelicula, devido a digitalizacao do cinema.
\end{itemize}

\end{document}
\end{spverbatim}

\section{Conclusão}
Em conclusão apesar do trabalho ser feito com o objetivo de palavras cinematográficas, pode-se contudo usar palavras de qualquer área. 

\appendix
\section{Código do Flex}
\small
\begin{changemargin}{-4cm}{-4cm}
\lstinputlisting[language=C, showstringspaces=false, breaklines]{../src/ctts.l}
\end{changemargin}

\normalsize
\section{Código do Yacc}
\small
\begin{changemargin}{-4cm}{-4cm}
\lstinputlisting[language=C, showstringspaces=false, breaklines]{../src/ctts.y}
\end{changemargin}

\end{document}
