\documentclass{llncs}
\usepackage[utf8]{inputenc}
\usepackage{fancyvrb} 
\usepackage[portuguese]{babel}
\usepackage{ragged2e}
\usepackage{listings}
\usepackage{spverbatim}

\def\changemargin#1#2{\list{}{\rightmargin#2\leftmargin#1}\item[]}
\let\endchangemargin=\endlist

\begin{document} \mainmatter
\title{Cinematographic Text Treatment System}
\titlerunning{Cinematographic Text Treatment System}
\author{José Carlos Lima Martins A78821 \and
        Miguel Miranda Quaresma A77049}
\authorrunning{José Carlos Lima Martins A78821 \and
        Miguel Miranda Quaresma A77049}
\institute{                                                                
University of Minho, Department of  Informatics, Braga, Portugal\\
e-mail: \{a78821,a77049\}@alunos.uminho.pt
}

\maketitle

\justify

\begin{abstract}
Neste documento é apresentada várias decisões e processos tomados de modo a desenvolver um parser Flex/Yacc que processe textos tendo em conta um dicionário que respeite a gramática construída. Este processamento de textos deve identificar palavras do texto que estão presentes no dicionário e sublinha-las adicionar um footnote e no fim do texto criar um apêndice com as palavras encontradas e o seu significado, tornando este processamento um texto normal(\verb|.txt| por exemplo) num ficheiro \LaTeX(\verb|.tex|).
\end{abstract}

\section{Introdução}
Neste trabalho desenvolvemos um sistema que a partir de um dicionário de palavras cinematográficas processa ficheiros de texto criando um ficheiro \LaTeX em que as palavras que estão presentes no dicionário e no texto são sublinhadas, sendo criadas footnotes bem como um apêndice com as definições das mesmas. Apesar de o trabalho ser vocacionado para um dicionário de palavras cinematográficas, o mesmo pode ser aplicado a palavras de todas as áreas estando apenas dependente do que pretende o utilizador.

Em relação à estrutura do relatório, inicialmente será exposto, de uma forma aprofundada, o problema a resolver sendo, de seguida, apresentada uma solução para o mesmo. São depois ilustrados vários exemplos de teste, finalizando com uma conclusão relativamente aos resultados do trabalho.

\section{Análise e especificação}

\subsection{Descrição informal do problema}
No âmbito da realização de filmes, pretende-se criar um Dicionário de palavras utilizadas nesta área. Associado a cada palavra pretende-se ter: o significado; a designação comum em inglês; e uma lista de sinónimos.

O sistema de tratamento de textos cinematográficos, \textbf{CTTS}, deve receber como argumento o nome do ficheiro onde está o dicionário(que respeite a linguagem/sintáxe por nós definida) e uma lista de ficheiros onde se encontram textos a anotar.

O objectivo do \textbf{CTTS} é ler o dicionário, armazenando-o numa estrutura interna de maneira a que possa processar os ditos textos. Para isso deve, para cada uma das palavras do texto que estejam presentes no dicionário, assinalá-las(sublinhando) e associando-lhes uma nota de rodapé(footnote) com o termo em inglês correspondente. No fim do texto será gerado um apêndice com a lista de palavras encontradas e respectivos significados.

\subsection{Especificação dos Requisitos}
O sistema desenvolvido deve permitir o uso de dicionários referentes a diversas áreas do conhecimento, não se restringindo apenas à área cinematográfica, algo que deve ser garantido através da gramática desenvolvida. Por outro lado, dado a quantidade de termos possíveis o processamento do dicionário e subsequente consulta da estrutura usada para armazenar o mesmo devem ser realizados de maneira eficiente \textbf{i.e.} em tempo útil. A ferramenta deve ainda permitir o processamento de diversos ficheiros de texto ao mesmo tempo gerando um ficheiro \LaTeX para cada um.

\section{Concepção/desenho da Resolução}

\subsection{Gramática do Dicionário}
Com o intuito de criar um formato para os dicionários que devem ser utilizados para consulta foi criada um gramática para os mesmos, gramática esta que permite associar cada palavra ao seu significado, o termo em inglês e os seus sinónimos. 
A estrutura desta gramática é a seguinte:
\begin{lstlisting}[mathescape]
Dicionario -> LinhaDic LinhasDic '.' 
LinhasDic -> $\epsilon$ 
          | ';' LinhaDic LinhasDic 
LinhaDic -> Palavra ':' Significado ':' Palavra ':' '[' Sinonimos 
Sinonimos -> ']' 
          | Palavra ListaSin ']' 
ListaSin -> $\epsilon$ 
         | ',' Palavra ListaSin 
Palavra -> pal 
Significado -> str 
\end{lstlisting}

Na gramática definida a \textbf{LinhaDic} representa uma linha individual no dicionario \textbf{i.e} cada palavra bem como o seu significado, o termo em inglês e os seus sinónimos. Cada LinhaDic (seguindo a sintaxe da linguagem de programação C) é separada pelo token ``;'' sendo que a última linha deve terminar com um ``.''. Os parâmetros de uma linha são separados por ``:'' sendo que o primeiro é a palavra da qual queremos obter informação, o Significado é, como o nome indica, o seu significado, seguido do termo em inglês e por fim uma lista de sinónimos. Os sinónimos estão dentro de parenteses retos sendo separados por virgulas, como uma lista.

\subsection{Estruturas}
Por forma a permitir o processamento de vários textos com um dicionário é necessário guardar o mesmo em memória. Para tal foram criadas as seguintes estruturas:
\renewcommand{\labelitemi}{$\bullet$}
\begin{itemize}
\item Lista ligada que permite representar os sinónimos
\begin{lstlisting}
typedef struct sinonym {
    char *sinonym;
    struct sinonym *next;
}*Sinonym;
\end{lstlisting}
\item Lista ligada que permite representar as linhas do dicionário
\begin{lstlisting}
typedef struct singleTerm{
    char *term;
    char *definition;
    char *designationEN;
    int refCount;
    Sinonym sinonyms;
    struct singleTerm *next;
}*SingleTerm;
\end{lstlisting}
\end{itemize}

A inserção da variável \texttt{refCount} na estrutura permite saber, findo o processamento de cada texto, que palavras foram encontradas no texto, sendo que, após criado o apêndice, esse valor volta a ser colocado a 0. Este valor pode, em futuras implementações, ser usado para saber quantos ``matches''(referências) houve para cada palavra do dicionário.

\subsection{Parse dos ficheiros}
Com a gramática do dicionário desenvolvida bem como a estrutura usada para guardar os dados do mesmo falta implementar uma das partes mais importantes do projeto, realizar o parse dos ficheiros. Esta fase pode ser dividida em duas:
\begin{itemize}
\item Parse do ficheiro com o dicionário, processo durante o qual devem ser identificadas as diferentes partes da gramática, retornando o conteúdo (caso necessário) e o tipo para o Yacc. No Yacc, consoante a parte da gramática encontrada são realizadas certas ações nomeadamente guardar os valores na estrutura anteriomente refeirda.
\item Parse dos ficheiros com os textos, sublinhando e criando notas de rodapé para as palavras que sejam comuns ao dicionário e ao texto, criando ainda um apêndice com as palavras encontradas e respetivos significados.
\end{itemize}

É por causa disto que no ficheiro Flex existem duas condições de contexto, DICT e FILES. Antes de cada iniciação do Flex é chamado uma das funções parseDict ou parseFiles para que o Flex comece respetivamente dentro da condição de contexto DICT ou FILES. Sendo assim, dentro de cada condição é colocado as regras especificas para cada tipo de ficheiro(DICT-Dicionário e FILES-ficheiro a ``converter''), permitindo assim realizar o parse de vários ficheiros com estruturas diferentes como o mesmo ficheiro Flex.

\subsection{Underline e footnote}
Para cada palavra presente nos ficheiros processados, a ferramente realiza uma consulta da estrutura onde foi armazenado o dicionário e, caso seja encontrado o termo em causa, a palavra é sublinhada no ficheiro de output, sendo ainda criada uma nota de rodapé com a sua homóloga em inglês: 
\begin{verbatim}
    if((match=searchTerm(yytext)))
        underlineAnnotate(yytext,match);          
    else fprintf(yyout,"%s",yytext);

    ...

    void underlineAnnotate(char *termPT, char *termEN){
        fprintf(yyout,"\\underline{%s}\\footnote{%s}",termPT,termEN);
    }
\end{verbatim}

De realçar que a ferramenta reconhece palavras que estejam identificadas como sinónimos de um outro termo, permitindo obter um comportamento mais versátil da mesma(ferramenta): 
\begin{verbatim}
    while(dictAux && !termEN){ 
        if(!strcmpCInsensitive(dictAux->term,term))
            termEN=dictAux->designationEN;
        sinAux=dictAux->sinonyms;
        while(sinAux && !termEN){
            if(!strcmpCInsensitive(sinAux->sinonym,term))
            termEN=dictAux->designationEN;
            sinAux=sinAux->next;
        }
        dictAux=dictAux->next;
    }
\end{verbatim}


\subsection{Funcionalidades Adicionais}
Por forma a aumentar a utilidade da ferramenta desenvolvida foram implementadas algumas funcionalidades adicionais, nomeadamente:
\begin{itemize}
    \item Possibilidade de escolher nome de ficheiro output para cada ficheiro de texto processado através da flag \textbf{-o}
    \item Procura/comparação de termos sem ter em conta a grafia(maiúscula ou minúscula) de uma letra
\end{itemize}


\subsection{Apêndice}
Como já descrito anteriormente a partir da variável refCount da estrutura SingleTerm é possível saber se a palavra foi ou não encontrada no texto. Sendo assim basta verificar esse valor e caso seja maior que 0 imprime-se um item de uma lista com o nome do termo e a definição:
\begin{verbatim}
while(aux){
    if(aux->refCount>0){
        fprintf(f,"\\item %s $\\to$ Def: %s\n",aux->term,aux->definition);
        aux->refCount=0;
    }
    aux=aux->next;
}
\end{verbatim}
Contudo é antes disto imprimido para o ficheiro a indicação de iniciação do apêndice, criada uma secção Apendice e a abertura da lista:
\begin{verbatim}
fprintf(f,"\n\\appendix\n");
fprintf(f,"\\section{Apendice}\n");
fprintf(f,"\\begin{itemize}\n");
\end{verbatim}
No fim de gerar todos os elementos da lista, fecha-se a lista bem como o documento:
\begin{verbatim}
fprintf(f,"\\end{itemize}\n");
fprintf(f,"\n\\end{document}\n");
\end{verbatim}

\section{Codificação e Testes}
Foi criado uma Makefile de modo a compilar o código sendo que a ideia base é:
\begin{itemize}
    \item Gerar \verb|.c| do flex: \verb|flex ctts.l|
    \item Gerar \verb|.c| do yacc: \verb|yacc -d -v ctts.y|
    \item Gerar o executável: \verb|gcc -o ctts y.tab.c|
\end{itemize}

O executável pode ser usado de formas diferentes dependendo de como é chamado e que flags são passadas:
\begin{itemize}
    \item \verb|./ctts <FicheiroDicionario> <FicheiroARealizarParse1> <FicheiroArealizarParse2> ...|
    \item \verb|./ctts <FicheiroDicionario> <FicheiroARealizarParse> -o <NomeDeFicheiroDeSaida>|
    \item 
    \begin{verbatim}
./ctts <FicheiroDicionario> <FicheiroARealizarParse1> -o <NomeDeFicheiroDeSaida1> 
                        <FicheiroARealizarParse2> -o <NomeDeFicheiroDeSaida2> ...
    \end{verbatim}
\end{itemize}
É importante frisar que no primeiro caso os nomes dos ficheiros de output serão \verb|NomeDoFicheiroInput.tex| enquanto que no segundo caso será o \verb|NomeInseridoPeloUser.tex|. No terceiro caso, será parecido ao segundo, ou seja o nome de saída de ``FicheiroARealizarParse1'' é \verb|NomeDeFicheiroDeSaida1.tex| e o nome de saída de ``FicheiroARealizarParse2'' é \verb|NomeDeFicheiroDeSaida2.tex|.

\subsection{Exemplo de teste}
Comando executado: \verb|./ctts dict.txt tt.txt -o file|
\subsubsection{Input: tt.txt}
\begin{spverbatim}
Tornar-se um roteirista começa com o desenvolvimento do seu talento como escritor, e o melhor jeito de desenvolver o seu talento como escritor é praticando. O seu primeiro projeto não precisa ser um roteiro de cinema ou de TV. Você apenas precisa começar em algum lugar: ensaios, contos, peças, etc. A idéia é estar habituado com o processo de escrita - uma experiência surpreendentemente lenta e frustrante - e encontrar o seu estilo. Com o passar do tempo, você descobrirá em que gênero seu talento é melhor aproveitado: comédia, drama ou outro.
\end{spverbatim}

\subsubsection{Output: file.tex}
\begin{spverbatim}
\documentclass{article}
\usepackage[bottom]{footmisc}

\begin{document}
Tornar-se um roteirista começa com o desenvolvimento do seu talento como escritor, e o melhor jeito de desenvolver o seu talento como escritor é praticando. O seu primeiro projeto não precisa ser um             \underline{roteiro}\footnote{script} de \underline{cinema}\footnote{cinema} ou de TV. Você apenas precisa começar em algum lugar: ensaios, contos, peças, etc. A idéia é estar habituado com o processo de escrita - uma experiência surpreendentemente lenta e frustrante - e encontrar o seu estilo. Com o passar do tempo, você descobrirá em que gênero seu talento é melhor aproveitado: comédia, drama ou outro.

\appendix
\section{Apendice}
\begin{itemize}
\item roteiro $\to$ Def: a historia do filme escrita em papel. Com as falas e tudo que for pertinente para a composicao do filme.
\item quadro $\to$ Def: a imagem unica estatica do filme, e a menor unidade de um filme. Varias imagens (frames) geram ilusao de movimento. Em geral, o ritmo e 24 quadros por segundo. Hoje em dia nao e necessario mais gravar em pelicula, devido a digitalizacao do cinema.
\end{itemize}

\end{document}
\end{spverbatim}

\section{Conclusão}
Em conclusão apesar do trabalho ser feito com o objetivo de identificar palavras do domínio cinematográfico, permite o uso palavras de qualquer área. Para além disso o processamento das palavras é case insensitive ou seja, identifica as palavras sem fazer distinção entre letras idênticas ainda que com grafias distintas. É também permitido, com recurso a uma flag, decidir qual o nome de sáida de cada ficheiro. É, por fim, importante referir que as ferramentas usadas(Flex e Yacc) aliadas à linguagem de programação C, dão-nos um grande poder de decisão e de desenvolvimento, tornado a que o desenvolvimento da aplicação fosse efetuado de maneira mais rápida, simples e mais fácil de se construir.

\appendix
\section{Código do Flex}
\small
\begin{changemargin}{-4cm}{-4cm}
\lstinputlisting[language=C, showstringspaces=false, breaklines]{../src/ctts.l}
\end{changemargin}

\normalsize
\section{Código do Yacc}
\small
\begin{changemargin}{-4cm}{-4cm}
\lstinputlisting[language=C, showstringspaces=false, breaklines]{../src/ctts.y}
\end{changemargin}

\end{document}
